\documentclass[12pt,a4paper]{scrartcl}

%\usepackage{lua-visual-debug}

\usepackage{lipsum}

\usepackage[lmargin=2.5cm, rmargin=2.5cm, top=3cm, bottom=3cm]{geometry}

\usepackage[ngerman]{babel}
\usepackage[babel=true,german=quotes]{csquotes}

\usepackage{microtype}

% Mathepakete (unicode-math ersetzt amssymb, amsfonts etc.)
\usepackage{amsmath}
\usepackage{mathtools}
\usepackage{mleftright}
\usepackage{fixdif,derivative}
\usepackage{physics2}
\usephysicsmodule{ab.legacy}
%\usepackage{bm}

\usepackage{fontspec}

\usepackage[math-style=ISO, partial=upright]{unicode-math}
%\usepackage[default]{fontsetup}
\setmainfont{Linux Libertine O}
%\setmathfont{New Computer Modern Math}
\setmathfont[StylisticSet=1]{NewCMMath-Book}
\setsansfont{Linux Biolinum O}

\mathitalicsmode=1
\let\oldright\right
\protected\def\right{^^^^200b\oldright}
\let\oldfrac\frac
\protected\def\frac#1#2{\oldfrac{#1^^^^200b}{#2^^^^200b}}

\newcommand{\italcorr}{^^^^200b}
\let\oldrparen\)
\protected\def\){\italcorr\oldrparen}

% Einheiten
\usepackage{siunitx}
\sisetup{locale = DE}

\usepackage{graphicx}
\usepackage{hyperref}
\hypersetup{
	colorlinks=true,
	linkcolor=blue,
	urlcolor=blue,
	citecolor=blue,
	pdftitle=Throne-of-the-Seas,
	pdfauthor=JM
}

\usepackage{mdframed}
\usepackage{xparse}
\usepackage[table]{xcolor}
\usepackage{tabularx}

\usepackage{tabularray}
\UseTblrLibrary{siunitx, booktabs}
\usepackage{tikz}
%\usepackage{pst-all}
\usepackage{romannum}
\AtBeginDocument{\pagenumbering{arabic}}
\usepackage{epsdice}
\usepackage{wrapfig}



\usepackage{graphicx}
\usepackage{hyperref}
\hypersetup{
    colorlinks=true,
    linkcolor=blue,
    urlcolor=blue,
    citecolor=blue,
    pdftitle=Throne of the Seas,
    pdfauthor=Julian Molt
}

\newcommand{\exphead}[1]{\textsf{\textbf{#1}}}

\setlength{\textfloatsep}{5pt} % Abstand zwischen Floats und Text
\setlength{\floatsep}{5pt}     % Abstand zwischen zwei Floats
\setlength{\intextsep}{5pt}    % Abstand für Floats, die im Text stehen


%Erklärkästchen
\newmdenv[
    linewidth=1pt,
    topline=false,
    bottomline=false,
    leftline=true,
    rightline=false,
    linecolor=lightgray,
    innerrightmargin=0,
    innerleftmargin=1em,
    innertopmargin=0.5em,
    innerbottommargin=0.5em,
    skipabove=1em,
    skipbelow=1em
]{expbox}

\title{\textsc{Throne of the Seas}}
\author{Lukas Schreiner, Johannes Philippin, Julian Molt}
\date{}

% Spaltenfarben
\newcolumntype{g}{>{\columncolor[HTML]{50C878}} c} % Grün
\newcolumntype{y}{>{\columncolor[HTML]{FFC800}} c} % Gelb
\newcolumntype{R}{>{\columncolor[HTML]{F75D59}} c} % Rot

\definecolor{Grün}{HTML}{50C878}
\definecolor{Gelb}{HTML}{FFC800}
\definecolor{Rot}{HTML}{F75D59}

\newcommand{\deph}[1]{\textsc{\textbf{#1}}} %Define Phrase

\begin{document}

\maketitle

\tableofcontents

\newpage

\section{Spielanleitung}

Ein \deph{Zug} ist jede Permutation der \deph{Aktionen} \emph{Segeln}, \emph{Handeln}, \emph{Schiff kaufen}, \emph{Tauschen}, \emph{Bergen} und \emph{Kämpfen}.

\subsection{Segeln}

Nach der Aktion \emph{Segeln} kann man stets handeln, tauschen, bergen oder kämpfen.\par
\noindent Jedes Feld, das auf dem Weg zum Zielort durchfahren wird, muss ein Meeresfeld sein.

\bigskip

\noindent Jeder Schiffstyp besitzt Standardwerte, die \deph{Segelstärken}, die in \autoref{sbscSchiffe} aufgeführt sind. Jede Runde wird nun eine neue Windkarte
aufgedeckt, die \deph{Windstärken} zwischen \(-4\) und \(+4\) in jede Richtung (N, NO, O, SO, S, SW, W, NW) bringen. Durch Addition
wird die höchstmögliche Anzahl an Kästchen, die in eine Richtung zurückgelegt werden kann, errechnet. Sie wird \deph{Windeseifer} genannt.
Nun gibt es mehrere Möglichkeiten, wie man sich bewegen kann:

\begin{enumerate}
    \item In eine Richtung segeln
    \item Auf der Stelle die Richtung ändern (wenden)
    \item In eine Richtung segeln und dann wenden
    \item In eine Richtung segeln, dann wenden und weiter, in eine beliebige Richtung segeln
\end{enumerate}

\begin{expbox}
    \exphead{In eine Richtung segeln}
    \medskip

    \noindent Hierfür wählt man erst eine der \num{8} möglichen Himmelsrichtungen, ermittelt die Windeseifer und setzt sein Schiff dann maximal so viele
    Felder in diese Richtung weiter.\par
    Die Aktion \emph{Segeln} ist beendet.
\end{expbox}

\begin{expbox}
    \exphead{Auf der Stelle wenden}
    \medskip

    \noindent Man sucht sich die Segel- und Windstärke aus, die die größte Windeseifer ergeben und kann sich dann in \textbf{\qty{90}{\degree}} Schritten % Boldfont funktioniert mit qty nicht
    auf der Stelle wenden. Eine Wendung kostet den Parameter \textbf{Wendigkeit} an Windeseifer. Die Windeseifer darf wenigstens \num{0} werden.\par
    Die Aktion \emph{Segeln} ist beendet.
\end{expbox}

\begin{expbox}
    \exphead{Segeln, dann wenden}
    \medskip

    \noindent Zuerst muss man wissen, dass man diesen Aktion ausführen will. Dann überlegt man sich, wie oft man nach dem
    Segeln wenden möchte. Die Windeseifer ergibt sich nun eindeutig aus der Segelrichtung vor Wendebeginn, im Gegensatz zu
    \enquote{Auf der Stelle wenden}. Die Anzahl an Wendungen, multipliziert mit dem Parameter Wendigkeit und dann von der Windeseifer subtrahiert ergibt die Windeseifer, welche nun zum Segeln in eine Richtung zur Verfügung steht. Jetzt kann man also
    segeln und dann wie überlegt wenden.\par
    Beispiel: \(\text{Segelstärke} = 4, \text{Windstärke} = 3, \text{Windeseifer} = 4 + 3 = 7, \text{Wendigkeit} = 2\)
    Möchte man nun \(1\times\) wenden verbraucht man dafür \num{2} Windeseifer, kann also davor nur noch \num{5} Felder in eine Richtung fahren.
    Oder man könnte sich auch um \qty{180}{\degree} drehen, also \(3\times\) und dafür \num{6} Windeseifer benötigen, und somit davor noch höchstens
    \num{1} Feld weit fahren.\par
    Die Aktion \emph{Segeln} ist beendet.
\end{expbox}

\begin{expbox}
    \exphead{Segeln, wenden, dann weitersegeln}
    \medskip

    \noindent Dieser Aktion funktioniert exakt wie \enquote{Segeln, dann wenden}, nur dass nach Beendigung des Wendevorgangs in jede Richtung %„“
    maximal so viel weitergesegelt werden kann, wie es der Parameter \deph{Ruderstärke} angibt. Dies ist die Zahl, die durch
    einen Schrägstrich hinter der Segelstärke steht.\par
    Die Aktion \emph{Segeln} ist beendet.
\end{expbox}

\subsection{Handeln}

Gehandelt wird an den Häfen der Inseln. Ein Schiff kann an einem Hafen nur Handel betreiben, wenn es direkt an ihm liegt.
Diagonale Felder gelten nicht.\par
Jeder Hafen hat eine \deph{Friedenszone}, in der keine Angriffe stattfinden können.\par
%Die Dingensens entsprechend einfärben, so wie auch in der Tabelle.
An einem Hafen besteht für je eine Ware ein \textcolor{Grün}{\deph{Warenüberschuss}}, eine \textcolor{Rot}{\deph{Warenkrise}} und für die restlichen Waren ein
\textcolor{Gelb}{\deph{Warenmangel}}. An einem Hafen können nur Waren, für die ein Warenüberschuss gilt, gekauft werden. Verkauft werden können jedoch
alle Waren, deren Preise sich entsprechend ihrer Nachfragelage an diesem Hafen ergeben.

\subsection{Schiffe kaufen}

Um ein Schiff kaufen zu können, benötigt man ein Schiff, das alle Waren, die für den Bau und Erwerb des Schiffes benötigt werden, auf einmal lagern kann.

\subsection{Tauschen}

\subsection{Bergen}\label{sbscBergen}

Gesunkene Schiffe und Schätze können geborgen werden. Dies wird durch \deph{Beuteplättchen} angezeigt. Um eine Bergungsaktion zu beginnen muss sich das bergende Schiff mit mindestens einem Feld auf dem Beuteplättchen befinden. Ist das Besatzungsmitglied \textit{Taucher} an Bord, kann die dort liegende Beute an Bord genommen werden, falls Frachtraum verfügbar ist.

\subsection{Besatzung}

Jedes Besatzungsmitglied darf höchstens einmal auf einem Schiff eingesetzt werden.

\subsection{Kämpfen}

\subsubsection{Feuergefecht}

Feuergefechte können zwischen jeglichen Schiffen stattfinden, also auch der eigenen Flotte, wenn man einen Grund
dafür sehen sollte.

\begin{expbox}
    \exphead{Reichweite}
    \medskip

    \noindent Eine Kanone hat eine \deph{Reichweite} von \textbf{zwei} zum eigenen Schiff anliegenden Feldern. Dies gilt nur für die Richtungen,
    in die die Kanonen ausgerichtet sind, also Bug, Heck, Backbord und Steuerbord. Nur diagonal kann also nicht geschossen werden.
\end{expbox}

\begin{expbox}
	\exphead{Teilseite}
	\medskip

	\noindent Eine Teilseite bezeichnet den Teil einer der vier kampffähigen Seiten eines Schiffes, der genau ein Feld auf dem Spielfeld lang ist. Demnach ist der Bug und das Heck immer eine Teilseite. Die Klasse eines Schiffes gibt Aufschluss darüber, aus wie vielen Teilseiten eine Seite höchstens besteht. So bestehen Back- und Steuerbord eines Schlachtkahns der Klasse \Romannum{3} aus jeweils \num{3} Teilseiten.
\end{expbox}

\begin{expbox}
    \exphead{Munition}
    \medskip

    \noindent Um einen Schuss abzugeben, benötigt man pro Kanone eine Munitionseinheit. Gefeuert wird immer pro Teilseite, sodass mehrere Einheiten verschossen werden. Munition muss gekauft werden. Schrotmunition macht einfachen Schaden, Kugelmunition doppelten.
\end{expbox}

\begin{expbox}
	\exphead{Distanzrang}
	\medskip

	\noindent Schiffe können sich nur in ein Feuergefecht begeben, wenn mindestens ein Distanzrang 0 existiert. Da Schiffe der \hyperref[data]{Klassen} \Romannum{2} \& \Romannum{3} \num{2} oder \num{3} Felder lang sind, müssen manche Kanonen bei diesen langen Schiffen diagonal schießen. Um dies darzustellen gibt es bestimmte Regeln, falls sich nur eine oder zwei Teilseiten direkt gegenüberstehen. Dazu wird der Distanzrang eingeführt, welcher sich in drei Stufen untergliedert. Er ergibt sich wie folgt:
	\begin{description}
		\item[Rang \(\symbf{0}\):] Die Verschiebung zwischen einer Teilseite, die eine Salve abgeben soll und der nächstgelegenen Teilseite des Ziels beträgt \num{0} Felder.
		Hier gilt volle Trefferwahrscheinlichkeit und bei den Werten \epsdice{3} \epsdice{4} \epsdice{5} \epsdice{6} wird getroffen.
		\item[Rang \(\symbf{1}\):] Die Verschiebung zwischen einer Teilseite, die eine Salve abgeben soll und der nächstgelegenen Teilseite des Ziels beträgt \num{1} Feld.
		Hier gilt verringerte Trefferwahrscheinlichkeit und bei den Werten \epsdice{4} \epsdice{5} \epsdice{6} wird getroffen.
		\item[Rang \(\symbf{2}\):] Die Verschiebung zwischen einer Teilseite, die eine Salve abgeben soll und der nächstgelegenen Teilseite des Ziels beträgt \num{2} Felder.
		Hier gilt die geringste Trefferwahrscheinlichkeit und nur bei den Werten \epsdice{5} oder \epsdice{6} wird getroffen.
	\end{description}
\end{expbox}

\begin{expbox}
    \exphead{Schaden ermitteln}
    \medskip

	\noindent Eine Breitseite besteht aus mehreren Salven, wobei eine Salve das Schießen einzelner Teilseiten repräsentiert.
    \begin{description}
    	\item[Kanonenschaden:] Dieser Schaden ergibt sich als Summe aus allen Werten, die Schaden für eine einzelne Kanone erhöhen können. Das sind die Art der Kanone selbst und die genutzte Munition. Hypothetischer Schaden, der als Grundlage der Berechnung für den Salvenschaden (den tatsächlichen Schaden) dient.
    	\item[Salvenschaden:] Eine Salve ist der Akt des Schießens einer Teilseite
    	\[
    	\text{Schaden} = \begin{cases}
    		0 & \text{Salve trifft nicht}\\
    		\text{Summe aller Kanonenschäden dieser Teilseite.} & \text{Salve trifft}
    	\end{cases}
    	\]
    	\item[Gesamtschaden:] Dies ist die Summe aller Salvenschäden.
    \end{description}
    Jetzt wird der Gesamtschaden von der Wandstärke der beschossenen Seite abgezogen. Fällt die Wandstärke einer Seite auf \num{0}, ist diese Breitseite zerstört, feuerunfähig und kann nur durch Reparatur (\autoref{sbsbscReparatur}) wieder einsatzfähig gemacht werden.
    Fällt die Gesamtwandstärke auf 0 ist das Schiff zerstört und kann nicht mehr repariert werden. Das Schiff wird dann mit einem Beuteplättchen, das zur Größe des entsprechenden Schiffes passt, unterlegt. Damit kann seine Ladung
    geborgen werden (\autoref{sbscBergen}).

    Die Aktion \emph{Kämpfen} ist beendet.
\end{expbox}

%\begin{expbox}
%   \exphead{Sonderfall bei teilweise überlappenden Seiten}
%    \medskip

%    \noindent Da Schiffe der Klassen \Romannum{2} \& \Romannum{3} zwei oder drei Felder lang sind, müssten Teile der Kanonen bei den längeren
%    Schiffen diagonal schießen. Um dies nachzuempfinden gibt es Sonderregeln, falls sich nur eine oder zwei Teilseiten direkt gegenüberstehen (sog. direkte Teilseiten). In diesen Fällen wird der mögliche Schaden der Teilseiten bestimmt.
%    Für den Schaden einer Teilseite, welche an der direkten anliegt (den 1. anliegenden Teilseite[n]), gilt eine verringerte
%    Trefferwahrscheinlichkeit, sodass nur bei folgenden Würfelwerten getroffen wird: \epsdice{4} \epsdice{5} \epsdice{6}.
%    Für die daran liegende Teilseite (2. anliegende Teilseite) gilt eine noch geringere Trefferwahrscheinlichkeit, sodass nur bei
%    \epsdice{5} oder \epsdice{6} getroffen wird. Für die direkte(n) Teilseite(n) wird weiterhin bei den Werten
%    \epsdice{3} \epsdice{4} \epsdice{5} \epsdice{6} getroffen.
%\end{expbox}

\subsubsection{Entern}

\subsubsection{Schiffsreparatur}\label{sbsbscReparatur}

Sobald ein Schiff beschädigt wurde kann es an einem Hafen repariert werden. Wird ein Schiff repariert, werden alle Wandstärken auf das Maximum erhöht. Der Reparaturpreis beträgt ein Viertel des Kaufpreises des Schiffes.

\newpage

\subsection{Ereigniskarten}

Jede Runde wird eine neue Ereigniskarte aufgedeckt. Eine Ereigniskarte kann nur ein Schiff betreffen, was meistens der Fall ist, oder kartenweiten Einfluss haben. Um zu bestimmen, welches Schiff das Ereignis betrifft, wird jedem Schiff bei Erstellung/Kauf eine fortlaufende Zahl zugewiesen. Durch würfeln mit einem Würfel für den \[\text{Anzahl Seiten Würfel} \leqslant \text{Anzahl Schiffe}\] gilt, wird bestimmt, welches Schiff getroffen wird.

Mögliche Ereignisse sind:

\begin{expbox}
	\exphead{Flaute}
	\medskip

	\noindent \textit{Die Segel erschlaffen, es ist Zeit das Schiff auf Vordermann zu bringen und die Rumfässer zu leeren: Eine Flaute macht sich breit.}\smallskip\\
	Ein ausgewürfeltes Schiff kann in dieser Runde nicht die Aktion \emph{Segeln} wählen.
\end{expbox}

\begin{expbox}
	\exphead{Segelschaden}
	\medskip

	\noindent \textit{Eine kräftige Böe fährt ins Hauptsegel -- zu kräftig. Die Matrosen können nur dabei zusehen, wie der Stoff zu Boden geht. Von Flüchen durchdrungene Luft kennzeichnet die Flickarbeiten. Sobald sie fertig sind, müssen sie nur noch hoffen, dass sie es damit bis in den nächstbesten Hafen schaffen.}\smallskip\\
	Ein ausgewürfeltes Schiff kann in dieser Runde nur mit halber Geschwindigkeit fahren. Es wird stets abgerundet.
\end{expbox}

\begin{expbox}
	\exphead{Hurrikan}
	\medskip

	\noindent \textit{Am Horizont bildet sich eine rießige, finstere Wolkenkonstellation. Ein starker Wind und Regen kommen auf. Das Meer raut in zunehmender Geschwindigkeit auf. Ein Hurrikan nähert sich!}\smallskip\\
	Das betroffene Schiff wird auf das erwürfelte Feld versetzt. Dieses Ereignis hat keine Wirkung, wenn sich das Schiff an einem Hafen befindet.
\end{expbox}

\begin{expbox}
	\exphead{Skorbut}
	\medskip

	\noindent \textit{Es ist schon lange her, dass die Matrosen frische Südfrüchte verspeist haben.}\smallskip\\
	Für das betroffene Schiff darf ausgesucht werden, welches Besatzungsmitglied verloren wird. Durch die Abgabe von einmal Südfrüchten oder einmal Medizin, kann dieser Effekt nullifiziert werden.
\end{expbox}

\begin{expbox}
	\exphead{Meuterei}
	\medskip

	\noindent \textit{Die Besatzung ist nicht mit ihrem Kapitän einverstanden und versammelt sich lauthals vor der Kapitänskajüte.}\smallskip\\
	Die Fähigkeiten der gesamten Besatzung werden für zwei Runden ausgesetzt.
\end{expbox}

\begin{expbox}
	\exphead{Kraken}
	\medskip

	\noindent \textit{Gigantische Luftblasen zerbersten die Meeresoberfläche und düstere Tentakel halten Ausschau nach unachtsamen Matrosen.}\smallskip\\
	Ein \(7 \times 7\) Felder großes Feld Bei dieser Ereigniskarte wird zunächst einer der Quadranten A, B, C, D, in dem der Kraken auftaucht, durch einen W4 bestimmt. Dann verschiebt sich seine Position auf ein weiteres, zufällig ausgewähltes Feld, ausgewählt durch \num{2}W\num{20} erst für Breiten- und dann für den Längengrad innerhalb des Quadranten. Für jede Runde, in der ein Schiff in der Krakenzone ist, muss es sich ein Feuergefecht mit dem Kraken liefern. Der Kraken hat folgende Eigenschaften:
	\begin{description}
		\item[Leben (Gesamtwandstärke):] \num{30}
		\item[Gesamtschaden:] Ein Viertel der Summe aller Kanonenschäden des Schiffes.
	\end{description}
	In diesem Kampf kann mit jeder Teilseite geschossen werden. Hierbei wird der Distanzrang durch die Distanzringe angegeben, AUF denen eine Teilseite liegt. Von Innen nach Außen entsprechen die Distanzringe den Distanzrängen \num{0}, \num{1} und \num{2}. Der Kraken greift immer mit dem Distanzrang an, der durch den niedrigsten Distanzring angegeben wird.
\end{expbox}

\begin{expbox}
	\exphead{Pulverfassexplosion}
	\medskip

	\noindent \textit{Den Smutje mit der Ladungssicherung zu beauftragen war keine gute Idee: Eine alte Öllampe fällt zu Boden und rollt in Richtung eines undichten Pulverfasses.}\smallskip\\

\end{expbox}

\begin{expbox}
	\exphead{Schatz}
	\medskip

	\noindent Schiffe
	\begin{description}
		\item[Gute Schiffsplanken:] Diese können zu einem beliebiegen Zeitpunkt verwendet werden um das Schiff vollständig zu reparieren.
		\item[Das Horn des Qun:] \(+5\) Wind in jede Richtung.
		\item[Schwarzer Rum:] Ein Crewmitglied verdoppelt seine Werte für eine Runde.
		\item[Zielwasser:] Der Distanzrang beträgt sofort \num{0}.
		\item[Elmsfeuer:] Wird man von einer Ereigniskarte getroffen, kann diese abgewehrt werden.
	\end{description}
\end{expbox}

\section{Spielverlauf}\label{verlauf}

\subsection{Spielbeginn}

\subsection{Spielziel}

Erreicht ein Spieler eines der Spielziele, endet das Spiel sofort und dieser Spieler gewinnt den \textsc{Thron der See}.

\begin{expbox}
	\exphead{Flottenadmiral}
	\medskip

	\noindent \textit{Ein wagemutiger Kapitän hat sich eine mächtige Flotte aufgebaut. Damit hat er sich den Thron der See gesichert.}\smallskip\\
	Um die Flottengröße \(F\) zu ermitteln, wird die Klasse \(K\) jedes Schiffes, das zur Flotte \(\Phi\) gehört, addiert. Sobald die Flottengröße \num{15} erreicht, ist das Spielziel erreicht. \[ F = \sum_{i \in {\Phi}} K_i \geqslant 15 \implies \text{Sieg}\]
\end{expbox}

\begin{expbox}
	\exphead{Handelskapitän}
	\medskip

	\noindent \textit{Ein gewitzter Seefahrer}\smallskip\\

\end{expbox}

\section{Datenblätter}\label{data}

\subsection{Schiffe}\label{sbscSchiffe}

\newpage

\subsubsection{Schaluppe}

\begin{tikzpicture}
    % Draw the hull shape
    \draw[thick] (0,0) -- (1,-3) -- (1.5,-6) -- (0,-9) -- (-1.5,-6) -- (-1,-3) -- cycle;

    % Add numbers
    \node at (0, 0.5) {4/2}; % Front
    \node at (-2, -5) {1/1}; % Left
    \node at (2, -5) {1/1};  % Right
    \node at (0, -9.5) {1/1}; % Back
\end{tikzpicture}
\begin{tikzpicture}
    % Draw the hull shape using smooth curves
    \draw[thick]
        (0,0)
        .. controls (1.5,-3) and (1.5,-6) .. (0,-9) % Right side
        .. controls (-1.5,-6) and (-1.5,-3) .. (0,0) % Left side
        -- cycle;

    % Add numbers
    \node at (0, 0.5) {4/2}; % Front
    \node at (-2, -5) {1/1}; % Left
    \node at (2, -5) {1/1};  % Right
    \node at (0, -9.5) {1/1}; % Back
\end{tikzpicture}

\newpage

\subsubsection{Kanonenboot}

\begin{minipage}[t]{0.6\textwidth} % Left side for text
   \lipsum[1-3]
    \end{minipage}%
    \hfill
    \begin{minipage}[t]{0.35\textwidth} % Right side for the figure
    \vspace{00pt}
%\begin{pspicture}(0,0)(5,10)
%    %\psaxes{->}(0,0)(0,0)(5,9)
%    \psgrid(0,0)(0,0)(5,10)
%    \pcline[linewidth=1pt]{-}(1.487,1)(3.5,1)
%    \pscurve[linewidth=1pt]{-}(1.5,1)(1.4,1.2)(1.24,1.6)(1.14,2)(1.065,2.4)(1.04,2.6)(1,3)(1,3.05)
%    \pscurve[linewidth=1pt]{-}(3.5,1)(3.6,1.2)(3.76,1.6)(3.86,2)(3.935,2.4)(3.96,2.6)(4,3)(4,3.05)
%    \pcline[linewidth=1pt]{-}(1,3)(1,7)
%    \pcline[linewidth=1pt]{-}(4,3)(4,7)
%    \pscurve[linewidth=1pt]{-}(1,6.95)(1,7)(1.04,7.3)(1.1,7.55)(1.25,8)(1.44,8.4)(1.6,8.66)(1.84,9)(2,9.2)(2.3,9.52)(2.5,9.7)
%    \pscurve[linewidth=1pt]{-}(4,6.95)(4,7)(3.96,7.3)(3.9,7.55)(3.75,8)(3.56,8.4)(3.4,8.66)(3.16,9)(3,9.2)(2.7,9.52)(2.5,9.7)
%\end{pspicture}
\begin{tikzpicture}
    \draw[thick] plot[smooth] coordinates {(1.487,1)(3.5,1)};
    \draw[thick] plot[smooth] coordinates {(1.5,1)(1.4,1.2)(1.24,1.6)(1.14,2)(1.065,2.4)(1.04,2.6)(1,3)(1,3.05)};
    \draw[thick] plot[smooth] coordinates {(3.5,1)(3.6,1.2)(3.76,1.6)(3.86,2)(3.935,2.4)(3.96,2.6)(4,3)(4,3.05)};
    \draw[thick] plot[smooth] coordinates {(1,3)(1,7)};
    \draw[thick] plot[smooth] coordinates {(4,3)(4,7)};
    \draw[thick] plot[smooth] coordinates {(1,6.95)(1,7)(1.04,7.3)(1.1,7.55)(1.25,8)(1.44,8.4)(1.6,8.66)(1.84,9)(2,9.2)(2.3,9.52)(2.5,9.7)};
    \draw[thick] plot[smooth] coordinates {(4,6.95)(4,7)(3.96,7.3)(3.9,7.55)(3.75,8)(3.56,8.4)(3.4,8.66)(3.16,9)(3,9.2)(2.7,9.52)(2.5,9.7)};
\end{tikzpicture}
\end{minipage}

\newpage

\subsubsection{Galeone}

\newpage

\subsubsection{Fregatte}

\newpage

\subsubsection{Frachtschiff}

\newpage

\subsubsection{Schlachtkahn}

\newpage

\subsection{Schiffskauf -- Übersicht}

\begin{table}[!h]
	\centering
	\begin{tblr}{
			colspec={Q[l]*{4}{Q[si={table-format=1.0},c]}Q[si={table-format=3.0},c,Grün]*{2}{Q[si={table-format=2.0},c]}},
			rowspec={*{7}{Q}},
			row{1} = {font=\bfseries, guard}
			}
		\toprule
		Schiff & Holz & Leinen & Eisen & Gold & Preis & Rep.-Preis & Benötigtes Lager \\
		\midrule
		Schaluppe    & 2 & 1 & 1 & {--} & 30  & 15 & 4  \\
		Kanonenboot  & 1 & 1 & 2 & {--} & 42  & 21 & 4  \\
		Galeone      & 2 & 1 & 3 & {--} & 58  & 29 & 6  \\
		Fregatte     & 1 & 1 & 4 & {--} & 70  & 35 & 6  \\
		Frachtschiff & 5 & 2 & 3 & {--} & 76  & 38 & 10 \\
		Schlachtkahn & 8 & 4 & 5 & 1    & 164 & 82 & 18 \\
		\bottomrule
	\end{tblr}
\end{table}

\subsection{Besatzung}

\begin{table}[!h]
	\centering
	\begin{tblr}{
			colspec={Q[l]Q[si={table-format=1.0},c]Q[si={table-format=2.0},c]Q[l]},
			rowspec={*{10}{Q}},
			row{1} = {font=\bfseries, guard}
			}
		\toprule
		Mitglied     & Leben & Preis & Effekt/Beschreibung. \\ \midrule
		Steuermann   & 4 & 75 & \(-1\) Wendekosten. \\
		Matrose      & 4 & 20 & \(+1\) Wind in die Richtung, in die der Wind dort negativ ist. \\
		Smutje       & 4 & 10 & \\
		Hexe         & 7 & 5  & Darf einen Würfel umdrehen. \\
		Kanonier     & 4 & 50 & \(+1\) Kanonenreichweite. \\
		Pirat        & 1 & 10 & \(+1\) Kleiner Würfel. \\
		Söldner      & 2 & 30 & \(+1\) Mittlerer Würfel. \\
		Seebär       & 4 & 70 & \(+1\) Großer Würfel. \\
		\midrule
		Taucher      & 2 & 50 & Erlaubt es Schätze und gesunkene Schiffe zu \hyperref[sbscBergen]{bergen}. \\
		\bottomrule
	\end{tblr}
\end{table}

\subsection{Kanonen}

\begin{table}[!h]
	\centering
	\begin{tblr}{
			colspec={Q[l]*{4}{Q[si={table-format=1.0},c]}},
			rowspec={*{5}{Q}},
			row{1} = {font=\bfseries, guard}
			}
		\toprule
		Kanone & Reichweite & Schaden & Roheisen & Goldnuggets\\
		\midrule
		Seeklinge        & 2 & 1 & {--} & {--} \\
		Rumpfbrecher     & 2 & 2 & 1 & {--} \\
		Dunkelkerker     & 3 & 1 & 1 & {--} \\
		Plankenschänder  & 3 & 2 & 1 & 1 \\
		\bottomrule
	\end{tblr}
\end{table}

\newpage

\newgeometry{lmargin=2cm, rmargin=2cm, top=3cm, bottom=3cm}

\subsection{Waren}

\begin{table}[!h]
	\centering
	\begin{tblr}{
			colspec={Q[l]Q[si={table-format=2.0}, c, Grün]Q[si={table-format=2.0}, c, Gelb]Q[si={table-format=2.0}, c, Rot]*{2}{Q[si={table-format=2.0},c]}},
			rowspec={*{9}{Q}},
			row{1} = {font=\bfseries, guard}
			}
		\toprule
		Ware & Kaufpreis & Mangel-Verkauf & Krise-Verkauf & Min. Profit & Max. Profit \\
		\midrule
		Dschungelholz & 2  & 6  & 8  & 4  & 6  \\
		Baumwolle     & 6  & 10 & 14 & 4  & 8  \\
		Südfrüchte    & 8  & 14 & 18 & 6  & 10 \\
		Leinentuch    & 12 & 18 & 24 & 6  & 12 \\
		Roheisen      & 14 & 22 & 28 & 8  & 14 \\
		Rumfass       & 20 & 28 & 36 & 8  & 16 \\
		Medizin       & 24 & 34 & 42 & 10 & 18 \\
		Goldnuggets   & 30 & 40 & 50 & 10 & 20 \\
		\bottomrule
	\end{tblr}
\end{table}

\restoregeometry

\end{document}